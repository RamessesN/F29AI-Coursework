\documentclass[12pt,a4paper]{article}

%----------------------------------------------------------------------------------------
% PACKAGES
%----------------------------------------------------------------------------------------
\usepackage[a4paper,margin=25mm]{geometry}
\usepackage{fontspec}
\setmainfont{Times New Roman}

\usepackage{amsmath}
\usepackage{graphicx,float,booktabs,array,multirow,url}
\usepackage{gensymb}
\usepackage[british]{babel}
\usepackage[square,numbers,sort&compress]{natbib}
\usepackage{caption,subcaption}
\usepackage{pgfplots}
\pgfplotsset{compat=1.9}

\usepackage[colorlinks=true,
            linkcolor=blue,
            citecolor=blue,
            urlcolor=blue]{hyperref}

\usepackage{minted}

\usepackage{fancyhdr}
\setlength{\headheight}{15pt}
\addtolength{\topmargin}{-2.5pt}

\usepackage{silence}
\WarningFilter{gensymb}{Not defining}

%----------------------------------------------------------------------------------------
% CUSTOM ABSTRACT FORMAT
%----------------------------------------------------------------------------------------
\makeatletter
\renewenvironment{abstract}{
    \begin{center}
        \large\bfseries \abstractname
    \end{center}
    \begin{quote}\small
}{
    \end{quote}
}
\makeatother

%----------------------------------------------------------------------------------------
% USER INFORMATION
%----------------------------------------------------------------------------------------
\newcommand{\studentonename}{Keqin ZHANG}
\newcommand{\studenttwoname}{Yuwei ZHAO}
\newcommand{\studentonenumber}{H00460395}
\newcommand{\studenttwonumber}{H00460398}
\newcommand{\labgroup}{OUC CW1 33}
\newcommand{\labdate}{2025-11-10}
\newcommand{\course}{F29AI - Artificial Intelligence}
\newcommand{\labtitle}{CW1 Report}

%----------------------------------------------------------------------------------------
% TITLE
%----------------------------------------------------------------------------------------
\title{
    \vspace{-1cm}
    \textbf{\labtitle}\\[0.3em]
    \Large \course
}
\author{
    \studentonename~(\studentonenumber)
    \quad
    \studenttwoname~(\studenttwonumber) \\[0.5em]
    \labgroup \quad \labdate
}
\date{}

%----------------------------------------------------------------------------------------
% PAGE STYLE
%----------------------------------------------------------------------------------------
\pagestyle{fancy}
\fancyhf{}
\fancyhead[L]{\labtitle}
\fancyhead[C]{\course}
\fancyhead[R]{\labgroup}
\fancyfoot[C]{\thepage}
\renewcommand{\headrulewidth}{0.4pt}
\renewcommand{\footrulewidth}{0pt}

\setlength{\parindent}{0em}
\setlength{\parskip}{0.75em}

%----------------------------------------------------------------------------------------
% DOCUMENT BODY
%----------------------------------------------------------------------------------------
\begin{document}

\maketitle

%----------------------------------------------------------------------------------------
\begin{abstract}
This report involves the solutions to the tasks outlined in Coursework 1 for the \texttt{F29AI - Artificial Intelligence} course.
The main objective of the coursework is to search algorithms and automated planning using \textit{PDDL}.

See Resources on \href{https://github.com/ZhangKeqin0307/couresework1.git}{git@github.com:ZhangKeqin0307/couresework1.git}.
\end{abstract}

%----------------------------------------------------------------------------------------
\section{Introduction}
% TODO: Write the introduction section.

%----------------------------------------------------------------------------------------
\section{Procedure}
%----------------------------------------------------------------------------------------
\subsection{Part 1 - Solving and Analyzing Sudoku with Search Algorithms}
\subsubsection{Part 1A}
\label{subsubsection:Part_1A}
\textbf{< Brief Description >} \\ [0.5em]
Assume that a \textit{Sudoku} consists of a $9 \times 9$ grid. 
The objective is to fill the grid with digits in such a way that each row, each column, and each of the 9 principal $3 \times 3$ sub-squares contains all of the digits from 1 to 9. 
An approach to an intelligent sudoku solver is needed.

\begin{itemize}
    \item Define Sudoku formally as a constraint satisfaction problem.
    \item What are the variables, domains, and constraints?
    \item Discuss the time complexity of brute-force search vs. backtracking in Sudoku.
\end{itemize}

\textbf{< Solution >} \\ [0.5em]
A CSP(constraint satisfaction problem) should involve the following three components: Variables, Domains and Constraints.
Therefore, we can define the Sudoku problem as follows:
\[
\text{Sudoku} = \langle V, D, C \rangle
\]
where
\begin{enumerate}
    \item $V$: The set of 81 variables, $V = \{V_{i,j} \mid i,j \in \{1, 2, \ldots, 9\}\}$.
    
    \item $D$: The domain $D_{i,j}$ for each variable $V_{i,j}$ is defined as:
    \begin{itemize}
        \item $D_{i,j} = \{k\}$, if $V_{i,j}$ is a given cell with value $k$.
        \item $D_{i,j} = \{1, 2, \ldots, 9\}$, if $V_{i,j}$ is an empty cell.
    \end{itemize}

    \item $C$: The set of 27 "All-Different" constraints:
    \begin{itemize}
        \item $C_{\text{row}}$: For each row $i$, all variables $V_{i, 1}, V_{i, 2}, \ldots, V_{i, 9}$ must have different values.
        \item $C_{\text{col}}$: For each column $j$, all variables $V_{1, j}, V_{2, j}, \ldots, V_{9, j}$ must have different values.
        \item $C_{\text{subgrid}}$: For each $3 \times 3$ subgrid, all 9 variables within that subgrid must have different values.
    \end{itemize}
\end{enumerate}

Time Complexity Analysis:
\begin{itemize}
    \item Brute-force Search Algorithm: \\
    For each of the $k$ spaces, there are 9 possible choices of numbers. This results in a total of $9 \times 9 \times \ldots \times 9$ (k times) combinations.
    Therefore, the time complexity of the brute-force search algorithm is $O(9^k)$.
    When the worst-case scenario occurs, the algorithm needs to explore all possible combinations, leading to the $O(9^{81})$ time complexity.

    \item Backtracking Search Algorithm: \\
    It checks the validity of constraints (row, column, and $3 \times 3$ sub-grid) \textbf{immediately} after assigning a number to a cell.
    If a conflict is detected (i.e., the current partial solution is invalid), the algorithm recursively "backtracks" to the previous step to try a different number. 
    This process effectively \textbf{prunes} large sub-trees of the search space that are known to be invalid.
    
    While the theoretical worst-case time complexity remains $O(9^k)$ (similar to brute-force), the average-case performance is drastically faster. 
    This is because the \textit{effective branching factor} $b$ becomes significantly smaller than 9 ($b \ll 9$) as the constraints restrict the number of valid choices for each subsequent cell.    
\end{itemize}

\newpage
\subsubsection{Part 1B}
\textbf{< Brief Description >} \\ [0.5em]
Build an intelligent \textit{Sudoku} solver that accepts input puzzles from \textit{.txt} or \textit{.csv} files. 
The program displays the solved puzzle in a clean format while printing relevant performance metrics, 
and concludes with a comparison against the \textit{A*} search algorithm.

\textbf{< Solution >} \\ [0.5em]
\begin{table}[H]
    \centering
    \begin{tabular}{|l|l|}
        \hline
        \textbf{Language} & Python \\ \hline
        \textbf{Solver} & Backtracking with pruning \\ \hline
        \textbf{Input} & $9 \times 9$ CSV file (0 represents empty) \\ \hline
        \textbf{Output} & GUI display and relevant metrics printed \\ \hline
        \textbf{Testing} & Several \textit{Sudoku} puzzle tests \\ \hline
    \end{tabular}
    \caption{Preparation of Sudoku Solver Implementation}
    \label{table:preparation_of_sudoku}
\end{table}

\hyperref[item:sudoku_solver]{\textbf{Backend Part:}}
\begin{enumerate}
    \item \textbf{Problem abstraction \& CSP modelling} \\
    As stated in Part~\ref{subsubsection:Part_1A}, model the puzzle as a CSP:
    \[
    \textit{Sudoku} = \langle V, D, C \rangle
    \]
    where \(V\) is the 81 variables \(V_{i,j}\); \(D_{i,j}\) are domains and \(C\) are the row / column / $3 \times 3$ subgrid all-different constraints. 
    
    In code this is corresponding to:
    \begin{itemize}
        \item \texttt{sudoku\_solver.board} : a $9 \times 9$ integer matrix (0 means empty).
        \item \texttt{is\_valid(row,col,num)} : enforces the three constraint types for a tentative assignment.
    \end{itemize}

    \item \textbf{Input format preset} \\
    Implement robust input routines that accept \texttt{.csv} formatted $9 \times 9$ grids.
    It needs to read rows with comma separation, convert each token to int, accept 0 or blank as empty.

    In our code base this is the method \texttt{load\_from\_csv(filepath)}. Validate that the parsed grid has $9~\text{rows} \times 9~\text{columns}$, otherwise throw / report an error.

    \item \textbf{Core solver build} \\
    ...
    
\end{enumerate}

\hrulefill

\hyperref[item:sudoku_gui]{\textbf{Frontend Part:}}
\begin{enumerate}
    \item ...
\end{enumerate}


\subsection{Part 2 - Automated Planning}
\subsubsection{Part 2A: Modelling the Domain}
\subsubsection{Part 2B: Modelling the Problems}
\subsubsection{Part 2C: Extension}

%----------------------------------------------------------------------------------------
\section{Reflection and Analysis}

%----------------------------------------------------------------------------------------
\section{Conclusion}

\newpage

%----------------------------------------------------------------------------------------
% Source Code
%----------------------------------------------------------------------------------------
\pagestyle{empty}
\section{Source Code}
\label{section:source_code}
\begin{itemize}
    \item \textbf{Part1 - \textit{sudoku\_solver.py}}
    \label{item:sudoku_solver}
    \begin{minted}[fontsize=\footnotesize,breaklines,linenos,frame=single]{python}
import csv
import time

class SudokuSolver:
    def __init__(self):
        self.board = []
        # Relevant metrics
        self.steps = 0            # Total number of steps
        self.recursive_calls = 0  # Recursive calls
        self.backtracks = 0       # Backtracks
        self.start_time = 0
        self.execution_time = 0

    def load_from_csv(self, filepath):
        self.board = []
        try:
            with open(filepath, 'r', encoding='UTF-8') as f:
                reader = csv.reader(f)
                for row in reader:
                    # Converts the string to an integer, handling possible whitespace
                    index = [int(num.strip()) for num in row if num.strip().isdigit()]
                    if len(index) == 9:
                        self.board.append(index)
            
            if len(self.board) != 9:
                raise ValueError("Invalid row count in CSV")
            
            print(f"[Succeed] File loaded: {filepath}")
            return True
        except Exception as e:
            print(f"[Error] File loading failed: {e}")
            return False

    def is_valid(self, row, col, num):
        # Row check
        for x in range(9):
            if self.board[row][x] == num:
                return False

        # Column check
        for x in range(9):
            if self.board[x][col] == num:
                return False

        # 3x3 box check
        start_row = row - row % 3
        start_col = col - col % 3
        for i in range(3):
            for j in range(3):
                if self.board[i + start_row][j + start_col] == num:
                    return False
        
        return True

    def solve_algorithm(self):
        self.recursive_calls += 1
        
        for i in range(9):
            for j in range(9):
                if self.board[i][j] == 0:
                    for num in range(1, 10):
                        self.steps += 1
                        # If invalid, prune it
                        if self.is_valid(i, j, num):
                            self.board[i][j] = num
                            
                            if self.solve_algorithm():
                                return True
                            
                            # Backtracking
                            self.board[i][j] = 0
                            self.backtracks += 1
                    
                    return False
        return True

    def run_solver(self):
        self.steps = 0
        self.recursive_calls = 0
        self.backtracks = 0
        
        self.start_time = time.perf_counter()
        
        success = self.solve_algorithm()
        
        end_time = time.perf_counter()
        self.execution_time = (end_time - self.start_time) * 1000
        
        return success

    def print_metrics(self):
        print("\n")
        print(f"Execution Time: {self.execution_time:.4f} ms")
        print(f"Total Steps (Attempts): {self.steps}")
        print(f"Recursive Calls: {self.recursive_calls}")
        print(f"Backtracks: {self.backtracks}")
        print("\n")
    \end{minted}

    \item \textbf{Part1 - \textit{sudoku\_gui.py}}
    \label{item:sudoku_gui}
    \begin{minted}[fontsize=\footnotesize,breaklines,linenos,frame=single]{python}
import os
import sys

from PyQt6.QtWidgets import (
    QApplication, QMainWindow, QWidget,     QGridLayout, 
    QVBoxLayout,  QHBoxLayout, QPushButton, QLineEdit, 
    QFileDialog,  QLabel,      QFrame,      QMessageBox
)
from PyQt6.QtCore import Qt
from PyQt6.QtGui import QFont, QIntValidator

from utils import load_qss, base_path
from sudoku_solver import SudokuSolver

class SudokuCell(QLineEdit):
    def __init__(self):
        super().__init__()
        self.setFixedSize(50, 50)
        self.setAlignment(Qt.AlignmentFlag.AlignCenter)
        self.setFont(QFont("Arial", 20))
        self.setMaxLength(1)
        self.setValidator(QIntValidator(1, 9))

        default_style_path = os.path.join(base_path(), "css_styles", "default_style.qss")
        solved_style_path = os.path.join(base_path(), "css_styles", "solved_style.qss")

        self.default_style = load_qss(default_style_path)
        self.solved_style = load_qss(solved_style_path)

        self.setStyleSheet(self.default_style)

    def set_solved_style(self):
        self.setStyleSheet(self.solved_style)

    def reset_style(self):
        self.clear()
        self.setStyleSheet(self.default_style)


class SudokuMainWindow(QMainWindow):
    def __init__(self):
        super().__init__()
        self.setWindowTitle(None)
        self.setFixedSize(600, 750)

        self.setStyleSheet("background-color: #333333;")
        
        self.solver = SudokuSolver()
        self.cells = [[None for _ in range(9)] for _ in range(9)]

        central_widget = QWidget()
        self.setCentralWidget(central_widget)
        self.main_layout = QVBoxLayout(central_widget) # Veritical Layout
        self.main_layout.setContentsMargins(20, 20, 20, 20)
        self.main_layout.setSpacing(15)

        title = QLabel("Sudoku")
        title.setFont(QFont("Arial", 25, QFont.Weight.Bold))
        title.setAlignment(Qt.AlignmentFlag.AlignCenter)
        title.setStyleSheet("color: #DDDDDD")
        self.main_layout.addWidget(title)

        self.grid_panel()

        self.info_panel = QLabel("Sudoku Game")
        self.info_panel.setObjectName("info_panel")
        self.info_panel.setFont(QFont("Arial", 16))

        panel_style_path = os.path.join(base_path(), "css_styles", "panel_style.qss")

        self.info_panel.setStyleSheet(load_qss(panel_style_path))
        self.info_panel.setAlignment(Qt.AlignmentFlag.AlignCenter)
        self.main_layout.addWidget(self.info_panel)

        self.button_panel()

        self.center_window()

    def grid_panel(self):
        grid_container = QWidget()

        outer_layout = QGridLayout(grid_container)
        outer_layout.setSpacing(10)

        for br in range(3):
            for bc in range(3):
                block_frame = QFrame()
                block_frame.setStyleSheet("background-color: #DDDDDD; border-radius: 5px;") 
                inner_layout = QGridLayout(block_frame)
                inner_layout.setSpacing(2)
                inner_layout.setContentsMargins(2, 2, 2, 2)

                for i in range(3):
                    for j in range(3):
                        global_r = br * 3 + i
                        global_c = bc * 3 + j
                        
                        cell = SudokuCell()
                        self.cells[global_r][global_c] = cell
                        inner_layout.addWidget(cell, i, j)
                
                outer_layout.addWidget(block_frame, br, bc)

        self.main_layout.addWidget(
            grid_container, 
            alignment = Qt.AlignmentFlag.AlignCenter
        )

    def button_panel(self):
        btn_layout = QHBoxLayout()
        
        button_style_path = os.path.join(base_path(), "css_styles", "button_style.qss")

        btn_load = QPushButton("Load CSV")
        btn_load.setStyleSheet(load_qss(button_style_path))
        btn_load.clicked.connect(self.load_csv)
        
        btn_solve = QPushButton("Solve Now")
        btn_solve.setStyleSheet(load_qss(button_style_path))
        btn_solve.clicked.connect(self.solve_puzzle)
        
        btn_clear = QPushButton("Clear")
        btn_clear.setStyleSheet(load_qss(button_style_path))
        btn_clear.clicked.connect(self.clear_board)

        btn_layout.addWidget(btn_load)
        btn_layout.addWidget(btn_solve)
        btn_layout.addWidget(btn_clear)
        
        self.main_layout.addLayout(btn_layout)

    def load_csv(self):
        file_path, _ = QFileDialog.getOpenFileName(self, "Open CSV", base_path(), "CSV Files (*.csv)")
        if file_path:
            success = self.solver.load_from_csv(file_path)
            if success:
                self.update_ui_from_solver()
                self.info_panel.setText(f"File Loaded: {file_path.split('/')[-1]}")
            else:
                QMessageBox.critical(self, "Error", "File loaded failed")

    def update_ui_from_solver(self):
        for r in range(9):
            for c in range(9):
                val = self.solver.board[r][c]
                cell = self.cells[r][c]
                cell.reset_style()
                if val != 0:
                    cell.setText(str(val))
                else:
                    cell.setText("")

    def sync_gui_to_backend(self):
        current_board = []
        for r in range(9):
            row_data = []
            for c in range(9):
                text = self.cells[r][c].text()
                if text.isdigit():
                    row_data.append(int(text))
                else:
                    row_data.append(0)
            current_board.append(row_data)
        
        self.solver.board = current_board

    def solve_puzzle(self):
        try:
            self.sync_gui_to_backend()

            if not self.self_check_isvalid():
                raise ValueError("[WARNING] Contradiction appears")
            
            if not self.solver.run_solver():
                raise ValueError("[ERROR] The question is fucking difficult")
            
            for r in range(9):
                for c in range(9):
                    val = self.solver.board[r][c]
                    cell = self.cells[r][c]
                    cell.setText(str(val))
                    cell.set_solved_style()

            metrics = (
                f"🕒 Time: {self.solver.execution_time:.2f} ms  |  "
                f"🔢 Steps: {self.solver.steps}  |  "
                f"🔙 Backtracks: {self.solver.backtracks}"
            )
            self.info_panel.setText(metrics)
            self.solver.print_metrics()
            
        except Exception as e:
            print(e)
            self.info_panel.setText("Try again!")
            QMessageBox.critical(self, "Error", str(e))

    def self_check_isvalid(self):
        board = self.solver.board
        
        for r in range(9):
            for c in range(9):
                num = board[r][c]
                if num != 0:
                   
                    board[r][c] = 0 
                    if not self.solver.is_valid(r, c, num):
                        board[r][c] = num
                        return False
                    board[r][c] = num
        return True
    
    def center_window(self):
        screen = QApplication.primaryScreen()
        screen_geometry = screen.availableGeometry()
        window_geometry = self.frameGeometry()
        window_geometry.moveCenter(screen_geometry.center())
        self.move(window_geometry.topLeft())

    def clear_board(self):
        self.solver.board = [[0]*9 for _ in range(9)]
        for r in range(9):
            for c in range(9):
                self.cells[r][c].clear()
                self.cells[r][c].reset_style()
        self.info_panel.setText("Clear!")

if __name__ == "__main__":
    app = QApplication(sys.argv)
    
    app.setStyle("Fusion") 
    
    window = SudokuMainWindow()
    window.show()
    sys.exit(app.exec())
    \end{minted}

    \item \textbf{Part2A\&B - \textit{domain1.pddl}}
    \begin{minted}[fontsize=\footnotesize,breaklines,linenos,frame=single]{PDDL}
(define (domain lunar)
    (:requirements :strips :typing)

    ; -------------------------------
    ; Types
    ; -------------------------------
    
    (:types
        lander
        rover
        location
        sample
        data
        image scan - data
    )

    ; -------------------------------
    ; Predicates
    ; -------------------------------

    (:predicates
        ; Positions
        (lander-at ?l - lander ?loc - location)
        (at ?r - rover ?loc - location)
        
        ; Connectivity between surface locations
        (path ?from - location ?to - location)

        ; The association between rovers and their respective landers
        (assigned ?r - rover ?l - lander)
        
        ; Landing location
        (unplaced ?l - lander)

        ; Deployment status
        (deployed ?r - rover)
        
        ; If lander had not carried sample 
        (lander-free ?l - lander)
        
        ; Data tasks
        ; Data objectives
        (image-target ?img - image ?loc - location)
        (scan-target ?sc - scan ?loc - location)
        ; If data has been captured
        (taken ?d - data)
        ; If rover hold data
        (empty-memory ?r - rover)
        (holding-data ?r - rover ?d - data)
        ; If data has been transmitted to lander
        (transmitted ?d - data ?l - lander)
        
        ; Sample tasks
        ; Sample objectives
        (sample-at ?s - sample ?loc - location)
        ; If rover hold sample
        (holding-sample ?r - rover ?s - sample)
        ; Collection of sample
        (sample-picked-up ?s - sample)
        (sample-stored ?s - sample) 
        (stored ?s - sample ?l - lander)
    )

    ; -------------------------------
    ; Actions
    ; -------------------------------
    
    ; Landing location
    (:action choose-landing
        :parameters (?l - lander ?wp - location)
        :precondition
            (and (unplaced ?l))
        :effect
            (and
                (not (unplaced ?l))
                (lander-at ?l ?wp)
            )
    )

    ; Deploy rover
    (:action deploy
        :parameters (?r - rover ?l - lander ?loc - location)
        :precondition (and
            (assigned ?r ?l)
            (lander-at ?l ?loc)
            (not (deployed ?r))
        )
        :effect (and
            (deployed ?r)
            (at ?r ?loc)
        )
    )
    
    ;Retrieve rover
    (:action retrieve
        :parameters (?r - rover ?l - lander ?loc - location)
        :precondition (and
            (assigned ?r ?l)
            (deployed ?r)
            (at ?r ?loc)
            (lander-at ?l ?loc)
        )
        :effect (and
            (not (deployed ?r))
            (not (at ?r ?loc))
        )
    )
    
    ; Movement of rover
    (:action move
        :parameters (?r - rover ?from - location ?to - location)
        :precondition (and
            (deployed ?r)
            (at ?r ?from)
            (path ?from ?to)
        )
        :effect (and
            (at ?r ?to)
            (not (at ?r ?from))
        )
    )

    ; Take Image
    (:action take-image
        :parameters (?r - rover ?img - image ?loc - location)
        :precondition (and
            (deployed ?r)
            (at ?r ?loc)
            (image-target ?img ?loc)
            (empty-memory ?r)
            (not (taken ?img))
        )
        :effect (and
            (holding-data ?r ?img)
            (taken ?img)
            (not (empty-memory ?r))
        )
    )

    ; Perform Scan
    (:action scan
        :parameters (?r - rover ?sc - scan ?loc - location)
        :precondition (and
            (deployed ?r)
            (at ?r ?loc)
            (scan-target ?sc ?loc)
            (empty-memory ?r)
            (not (taken ?sc))
        )
        :effect (and
            (holding-data ?r ?sc)
            (taken ?sc)
            (not (empty-memory ?r))
        )
    )
    
    ; Transmit Data to Lander
    (:action transmit
        :parameters (?r - rover ?d - data ?l - lander)
        :precondition (and
            (assigned ?r ?l)
            (deployed ?r)
            (holding-data ?r ?d)
        )
        :effect (and
            (transmitted ?d ?l)
            (empty-memory ?r)
            (not (holding-data ?r ?d))
        )
    )
    
    ; Pick up Sample
    (:action pick-up-sample
        :parameters (?r - rover ?s - sample ?loc - location)
        :precondition (and
            (deployed ?r)
            (at ?r ?loc)
            (sample-at ?s ?loc)
            (not (sample-picked-up ?s))
        )
        :effect (and
            (holding-sample ?r ?s)
            (sample-picked-up ?s)
            (not (sample-at ?s ?loc))
        )
    )
    
    ; Store Sample in Lander
    (:action store-sample
        :parameters (?r - rover ?s - sample ?l - lander)
        :precondition (and
            (not (deployed ?r)) 
            (holding-sample ?r ?s)
            (assigned ?r ?l)
            (lander-free ?l)
        )
        :effect (and
            (stored ?s ?l)
            (sample-stored ?s)
            (not (holding-sample ?r ?s))
            (not (lander-free ?l))
        )
    )
)    
    \end{minted}

    \item \textbf{Part2A\&B - \textit{mission1.pddl}}
    \begin{minted}[fontsize=\footnotesize,breaklines,linenos,frame=single]{PDDL}
(define (problem lunar-mission-1)
    (:domain lunar)

    (:objects
        lander1 - lander
        rover1 - rover
        ; Map
        wp1 wp2 wp3 wp4 wp5 wp6 - location
        ; Data
        image5 - image
        scan3 - scan
        ; Sample
        sample1 - sample
    )

    (:init
        ; The association between rover1 and lander1
        (assigned rover1 lander1)
        
        ; Lander1 can land at any location
        (unplaced lander1)
        
        ; Rover1 has not been deployed
        (not (deployed rover1))
        
        ; Lander1 has not stored sample
        (lander-free lander1)
        
        ; Rover1 memory is empty
        (empty-memory rover1)
        
        ; Connectivity between surface locations(Figure 2)
        (path wp1 wp4)
        (path wp4 wp3)
        (path wp3 wp5)
        (path wp5 wp1)
        (path wp1 wp2)
        (path wp2 wp3)
        
        ; task objective
        (image-target image5 wp5)
        (scan-target scan3 wp3)
        (sample-at sample1 wp1)
    )

    (:goal
        (and
            ; Indicates that image5 has been captured
            (taken image5)
            
            ; Indicates that scan3 has been captured
            (taken scan3)
            
            ; Indicates that sample1 has been collected and brought back
            (sample-stored sample1) 
        )
    )
)
    \end{minted}

    \item \textbf{Part2A\&B - \textit{mission2.pddl}}
    \begin{minted}[fontsize=\footnotesize,breaklines,linenos,frame=single]{PDDL}
(define (problem lunar-mission-2)
    (:domain lunar)

    (:objects
        lander1 lander2 - lander
        rover1 rover2 - rover
        ; Map
        wp1 wp2 wp3 wp4 wp5 wp6 - location
        ; Data
        image2 image3 - image
        scan4 scan6 - scan
        ; Sample
        sample1 sample5 - sample
    )

    (:init
        ; The association between rovers and their respective landers
        (assigned rover1 lander1)
        (assigned rover2 lander2)

        ; Rover initial deployment states
        ; Rover1 starts deployed at wp2 (with lander2 at wp2)
        (not (unplaced lander1))
        (lander-at lander1 wp2)
        (at rover1 wp2)
        (deployed rover1)
        
        ; Initial status
        (lander-free lander1)
        (empty-memory rover1)
        
        ; Rover2 starts undeployed (lander2 unplaced ( to be chosen))
        (unplaced lander2)
        (not (deployed rover2))
        
        ; Initial status of 1
        (lander-free lander2)
        (empty-memory rover2)

        ; Connectivity between surface locations(Figure 2)
        (path wp1 wp2)
        (path wp2 wp1)
        (path wp2 wp3)
        (path wp3 wp5)
        (path wp5 wp3)
        (path wp5 wp6)
        (path wp6 wp4)
        (path wp2 wp4)
        (path wp4 wp2)

        ; task objective
        ; Image
        (image-target image2 wp2)
        (image-target image3 wp3)
        ; Scan
        (scan-target scan4 wp4)
        (scan-target scan6 wp6)
        ; Sample
        (sample-at sample1 wp1)
        (sample-at sample5 wp5)
    )

    (:goal
        (and
            ; Indicates that image2 and image3 have been captured
            (taken image2)
            (taken image3)
            
            ; Indicates that scan4 and scan6 have been captured
            (taken scan4)
            (taken scan6)

            ; Indicates that sample1 and sample5 have been collected(picked up and stored)
            (sample-stored sample1)
            (sample-stored sample5)
        )
    )
)
    \end{minted}

    \item \textbf{Part2C - \textit{domain2.pddl}}
    \begin{minted}[fontsize=\footnotesize,breaklines,linenos,frame=single]{PDDL}
(define (domain lunar)
    (:requirements :strips :typing)

    ; -------------------------------
    ; Types
    ; -------------------------------
    
    (:types
        lander
        rover
        location
        sample
        data
        image scan - data
        ; New:
        astronaut
        area
    )

    ; -------------------------------
    ; Predicates
    ; -------------------------------

    (:predicates
        ; positions
        (lander-at ?l - lander ?loc - location)
        (at ?r - rover ?loc - location)
        
        ; Connectivity between surface locations
        (path ?from - location ?to - location)

        ; The association between rovers and their respective landers
        (assigned ?r - rover ?l - lander)
        
        ; Landing location
        (unplaced ?l - lander)

        ; deployment status
        (deployed ?r - rover)
        
        ; If lander had not carried sample 
        (lander-free ?l - lander)
        
        ; Data tasks
        ; Data objectives
        (image-target ?img - image ?loc - location)
        (scan-target ?sc - scan ?loc - location)
        ; If data has been captured
        (taken ?d - data)
        ; If rover hold data
        (empty-memory ?r - rover)
        (holding-data ?r - rover ?d - data)
        ; If data has been transmitted to lander
        (transmitted ?d - data ?l - lander)
        
        ; Sample tasks
        ; Sample objectives
        (sample-at ?s - sample ?loc - location)
        ; If rover hold sample
        (holding-sample ?r - rover ?s - sample)
        ; Collection of sample
        (sample-picked-up ?s - sample)
        (sample-stored ?s - sample) 
        (stored ?s - sample ?l - lander)
        
        ; New: 
        ; The astronaut is located in which area of the lander
        (crew-in ?a - astronaut ?l - lander ?ar - area)
    )

    ; -------------------------------
    ; Actions
    ; -------------------------------
    
    ; New action: 
    ; Movement of astronaut in lander
    (:action move-crew
        :parameters (?a - astronaut ?l - lander ?from - area ?to - area)
        :precondition (crew-in ?a ?l ?from)
        :effect (and
            (crew-in ?a ?l ?to)
            (not (crew-in ?a ?l ?from))
        )
    )
    
    ; Landing location
    (:action choose-landing
        :parameters (?l - lander ?wp - location)
        :precondition
            (and (unplaced ?l))
        :effect
            (and
                (not (unplaced ?l))
                (lander-at ?l ?wp)
            )
    )
    
    ; Deploy rover
    (:action deploy
        :parameters (?r - rover ?l - lander ?loc - location ?a - astronaut)
        :precondition (and
            (assigned ?r ?l)
            (lander-at ?l ?loc)
            (not (deployed ?r))
            (crew-in ?a ?l docking-bay)
        )
        :effect (and
            (deployed ?r)
            (at ?r ?loc)
        )
    )
    
    ;Retrieve rover
    (:action retrieve
        :parameters (?r - rover ?l - lander ?loc - location ?a - astronaut)
        :precondition (and
            (assigned ?r ?l)
            (deployed ?r)
            (at ?r ?loc)
            (lander-at ?l ?loc)
            (crew-in ?a ?l docking-bay)
        )
        :effect (and
            (not (deployed ?r))
            (not (at ?r ?loc))
        )
    )
    
    ; Movement of rover
    (:action move
        :parameters (?r - rover ?from - location ?to - location)
        :precondition (and
            (deployed ?r)
            (at ?r ?from)
            (path ?from ?to)
        )
        :effect (and
            (at ?r ?to)
            (not (at ?r ?from))
        )
    )

    ; Take Image
    (:action take-image
        :parameters (?r - rover ?img - image ?loc - location)
        :precondition (and
            (deployed ?r)
            (at ?r ?loc)
            (image-target ?img ?loc)
            (empty-memory ?r)
            (not (taken ?img))
        )
        :effect (and
            (holding-data ?r ?img)
            (taken ?img)
            (not (empty-memory ?r))
        )
    )

    ; Perform Scan
    (:action scan
        :parameters (?r - rover ?sc - scan ?loc - location)
        :precondition (and
            (deployed ?r)
            (at ?r ?loc)
            (scan-target ?sc ?loc)
            (empty-memory ?r)
            (not (taken ?sc))
        )
        :effect (and
            (holding-data ?r ?sc)
            (taken ?sc)
            (not (empty-memory ?r))
        )
    )
    
    ; Transmit Data to Lander
    (:action transmit
        :parameters (?r - rover ?d - data ?l - lander ?a - astronaut)
        :precondition (and
            (assigned ?r ?l)
            (deployed ?r)
            (holding-data ?r ?d)
            (crew-in ?a ?l control-room)
        )
        :effect (and
            (transmitted ?d ?l)
            (empty-memory ?r)
            (not (holding-data ?r ?d))
        )
    )
    
    ; Pick up Sample
    (:action pick-up-sample
        :parameters (?r - rover ?s - sample ?loc - location)
        :precondition (and
            (deployed ?r)
            (at ?r ?loc)
            (sample-at ?s ?loc)
            (not (sample-picked-up ?s))
        )
        :effect (and
            (holding-sample ?r ?s)
            (sample-picked-up ?s)
            (not (sample-at ?s ?loc))
        )
    )
    
    ; Store Sample in Lander
    (:action store-sample
        :parameters (?r - rover ?s - sample ?l - lander ?a - astronaut)
        :precondition (and
            (not (deployed ?r)) 
            (holding-sample ?r ?s)
            (assigned ?r ?l)
            (lander-free ?l)
            (crew-in ?a ?l docking-bay)
        )
        :effect (and
            (stored ?s ?l)
            (sample-stored ?s)
            (not (holding-sample ?r ?s))
            (not (lander-free ?l))
        )
    )
) 
    \end{minted}

    \item \textbf{Part2C - \textit{mission3.pddl}}
    \begin{minted}[fontsize=\footnotesize,breaklines,linenos,frame=single]{PDDL}
(define (problem lunar-mission-2)
    (:domain lunar)

    (:objects
        lander1 lander2 - lander
        rover1 rover2 - rover
        ; Map
        wp1 wp2 wp3 wp4 wp5 wp6 - location
        ; Data
        image2 image3 - image
        scan4 scan6 - scan
        ; Sample
        sample1 sample5 - sample
        ; New:
        ; Area
        control-room docking-bay - area  
        ; People
        alice bob - astronaut

    )

    (:init
        ; New:
        ; Crew position
        (crew-in alice lander1 docking-bay)  
        (crew-in bob lander2 control-room)
        
        ; The association between rovers and their respective landers
        (assigned rover1 lander1)
        (assigned rover2 lander2)

        ; Rover initial deployment states
        ; Rover1 starts deployed at wp2 (with lander2)
        (not (unplaced lander1))
        (lander-at lander1 wp2)
        (at rover1 wp2)
        (deployed rover1)
        
        ; Initial status
        (lander-free lander1)
        (empty-memory rover1)
        
        ; Rover2 starts undeployed (lander2 unplaced ( to be chosen))
        (unplaced lander2)
        (not (deployed rover2))
        
        ; Initial status of 2
        (lander-free lander2)
        (empty-memory rover2)

        ; Connectivity between surface locations(Figure 3)
        (path wp1 wp2)
        (path wp2 wp1)
        (path wp2 wp3)
        (path wp3 wp5)
        (path wp5 wp3)
        (path wp5 wp6)
        (path wp6 wp4)
        (path wp2 wp4)
        (path wp4 wp2)

        ; task objective
        ; Image
        (image-target image2 wp2)
        (image-target image3 wp3)
        ; Scan
        (scan-target scan4 wp4)
        (scan-target scan6 wp6)
        ; Sample
        (sample-at sample1 wp1)
        (sample-at sample5 wp5)
    )

    (:goal
        (and
            ; Indicates that image2 and image3 have been captured
            (taken image2)
            (taken image3)
            
            ; Indicates that scan4 and scan6 have been captured
            (taken scan4)
            (taken scan6)

            ; Indicates that sample1 and sample5 have been collected(picked up and stored)
            (sample-stored sample1)
            (sample-stored sample5)
        )
    )
)
    \end{minted}
\end{itemize}

\newpage

%----------------------------------------------------------------------------------------
% Reference
%----------------------------------------------------------------------------------------
\section{Reference}
\label{section:reference}
\begin{verbatim}
@ CSP 相关
<https://en.wikipedia.org/wiki/Constraint_satisfaction_problem>
<https://www.geeksforgeeks.org/artificial-intelligence/constraint-satisfaction-problems-csp-in-artificial-intelligence/>

@ Brute-force
<https://en.wikipedia.org/wiki/Brute-force_search>
<https://www.geeksforgeeks.org/dsa/brute-force-approach-and-its-pros-and-cons/>

@ Backtracking
<https://en.wikipedia.org/wiki/Backtracking>
<https://www.geeksforgeeks.org/dsa/backtracking-algorithms/>

@ pyqt 相关
1. <https://www.riverbankcomputing.com/static/Docs/PyQt6/>
2. <https://www.runoob.com/python3/python-pyqt.html>
3. <https://blog.csdn.net/u012117917/article/details/41604711>
4. <https://zhuanlan.zhihu.com/p/390192953>

@ 
\end{verbatim}

\end{document}