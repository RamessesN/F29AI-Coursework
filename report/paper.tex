\documentclass[12pt,a4paper]{article}

%----------------------------------------------------------------------------------------
% PACKAGES
%----------------------------------------------------------------------------------------
\usepackage[a4paper,margin=25mm]{geometry}
\usepackage{fontspec}
\setmainfont{Times New Roman}

\usepackage{amsmath}
\usepackage{graphicx,float,booktabs,array,multirow,url}
\usepackage{gensymb}
\usepackage[british]{babel}
\usepackage[square,numbers,sort&compress]{natbib}
\usepackage{caption,subcaption}
\usepackage{pgfplots}
\pgfplotsset{compat=1.9}

\usepackage[colorlinks=true,
            linkcolor=blue,
            citecolor=blue,
            urlcolor=blue]{hyperref}

\usepackage{minted}

\usepackage{fancyhdr}
\setlength{\headheight}{15pt}
\addtolength{\topmargin}{-2.5pt}

\usepackage{silence}
\WarningFilter{gensymb}{Not defining}

%----------------------------------------------------------------------------------------
% CUSTOM ABSTRACT FORMAT
%----------------------------------------------------------------------------------------
\makeatletter
\renewenvironment{abstract}{
    \begin{center}
        \large\bfseries \abstractname
    \end{center}
    \begin{quote}\small
}{
    \end{quote}
}
\makeatother

%----------------------------------------------------------------------------------------
% USER INFORMATION
%----------------------------------------------------------------------------------------
\newcommand{\studentonename}{Keqin ZHANG}
\newcommand{\studenttwoname}{Yuwei ZHAO}
\newcommand{\studentonenumber}{H00460395}
\newcommand{\studenttwonumber}{H00460398}
\newcommand{\labgroup}{OUC CW1 33}
\newcommand{\labdate}{2025-11-10}
\newcommand{\course}{F29AI - Artificial Intelligence}
\newcommand{\labtitle}{Coursework1 Report}

%----------------------------------------------------------------------------------------
% TITLE
%----------------------------------------------------------------------------------------
\title{
    \vspace{-1cm}
    \textbf{\labtitle}\\[0.3em]
    \Large \course
}
\author{
    \studentonename~(\studentonenumber)
    \quad
    \studenttwoname~(\studenttwonumber) \\[0.5em]
    \labgroup \quad \labdate
}
\date{}

%----------------------------------------------------------------------------------------
% PAGE STYLE
%----------------------------------------------------------------------------------------
\pagestyle{fancy}
\fancyhf{}
\fancyhead[L]{\labtitle}
\fancyhead[R]{\labgroup}
\fancyfoot[C]{\thepage}
\renewcommand{\headrulewidth}{0.4pt}
\renewcommand{\footrulewidth}{0pt}

\setlength{\parindent}{0em}
\setlength{\parskip}{0.75em}

%----------------------------------------------------------------------------------------
% DOCUMENT BODY
%----------------------------------------------------------------------------------------
\begin{document}

\maketitle

%----------------------------------------------------------------------------------------
\begin{abstract}
This report involves the solutions to the tasks outlined in Coursework 1 for the \textit{F29AI - Artificial Intelligence course}.
The main objective of the coursework is to search algorithms and automated planning using $PDDL$.
\end{abstract}

%----------------------------------------------------------------------------------------
\section{Introduction}


%----------------------------------------------------------------------------------------
\section{Procedure}
%----------------------------------------------------------------------------------------
\subsection{Part 1 - Solving and Analyzing Sudoku with Search Algorithms}
\subsubsection{Part 1A}
\textbf{Solution:} \\
A CSP(constraint satisfaction problem) should involves the following three components: Variables, Domains and Constraints.
Therefore, we can define the Sudoku problem as follows:
\[
\text{Sudoku} = \langle V, D, C \rangle
\]
where
\[
\begin{array}{rl p{8cm}}
V &= \{V_{i,j} \mid i,j \in \{1,2,\ldots,9\}\}, 
  & variables representing each cell in the 9×9 grid; \\[0.4em]
D &= \{D_{i,j} = \{1,2,\ldots,9\} \mid i,j \in \{1,2,\ldots,9\}\}, 
  & domains of possible values for each cell; \\[0.4em]
C &= \{C_k \mid k \in \{1,2,\ldots,27\}\}, 
  & constraints enforcing unique values per row, column, and 3×3 box.
\end{array}
\]

\textbf{Time Complexity Analysis:}
\begin{itemize}
    \item Brute-force Search Algorithm: \\
    For each of the $k$ spaces, there are 9 possible choices of numbers. This results in a total of $9 \times 9 \times \ldots \times 9$ (k times) combinations.
    Therefore, the time complexity of the brute-force search algorithm is $O(9^k)$.
    When the worst-case scenario occurs, the algorithm needs to explore all possible combinations, leading to the $O(9^{81})$ time complexity.
    \item Backtracking Search Algorithm: \\
    
\end{itemize}

\subsubsection{Part 1B}

\subsection{Part 2 - Automated Planning}
\subsubsection{Part 2A: Modelling the Domain}
\subsubsection{Part 2B: Modelling the Problems}
\subsubsection{Part 2C: Extension}

%----------------------------------------------------------------------------------------
\section{Reflection and Analysis}

%----------------------------------------------------------------------------------------
\section{Conclusion}

\newpage

%----------------------------------------------------------------------------------------
% Source Code
%----------------------------------------------------------------------------------------

\end{document}